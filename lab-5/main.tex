\documentclass{article}

% Packages
\usepackage{geometry}
\usepackage{graphicx}
\usepackage{lipsum}
\usepackage{tcolorbox}
\usepackage[portuguese]{babel}
\usepackage[margin=4cm]{caption}

\tcbset{%any default parameters
    toprule=3mm,
    arc=2mm,
    bottomtitle=2mm
}

% Page setup
\geometry{a4paper, margin=2cm}

\begin{document}

\input{cover.tex}
\newpage

\thispagestyle{empty}
\tableofcontents
\newpage

\section{Introdução}

Durante o Laboratório 5, fomos encargados em trabalhar com o diodo zener. O diodo zener (também conhecido com diodo regulador de tensão), tem funcionamento semelhante à um diodo convencional, porém é projetado para conduzir em regime de condução inversa. \\

Mais sobre o diodo zener é apresentado na Secção \ref{zener}

\subsection{Objetivos}
Ao final do experiemento, devemos ser capazes de:


\begin{tcolorbox}
    \begin{enumerate}
        \large
        \item Realizar medições de corrente e tensão utilizando um multímetro;
        \item Analisar o efeito do diodo zener como regulador de tensão em fontes de alimentação e suas limitações;
        \item Analisar formas de onda utilizando o osciloscópio.
    \end{enumerate}
\end{tcolorbox}

\section{Experimento}

\subsection{Materiais e equipamentos necessários}

\begin{tcolorbox}
    \begin{itemize}
        \large
        \item 1 Osciloscópio;
        \item 2 Multímetros;
        \item 1 Módulo de fonte de CC;
        \item Conjunto de cabos de teste.
    \end{itemize}
\end{tcolorbox}

\subsection{Teoria}
\subsubsection{Diodo Zener}
\label{zener}

Os diodos Zener são projetados para operar de forma estável sob condições de ruptura reversa, ao contrário dos diodos comuns. Na curva característica de um diodo Zener, a área do lado esquerdo do gráfico (área de polarização reversa) é onde a tensão permanece quase constante, independentemente das mudanças na corrente após atingir a tensão Zener. Esta tensão de ruptura reversa é a tensão Zener. Essencialmente, a tensão permanece estável mesmo quando a corrente reversa aumenta, permitindo que o diodo atue como um regulador de tensão, protegendo o circuito contra sobretensão. Quando a corrente flui nesta região de ruptura, o diodo Zener mantém uma tensão de saída estável, o que é ideal para aplicações de regulação de tensão.

\begin{figure}[h!]
    \centering
    \includegraphics[width=8cm]{images/diodo-zener.png}
    \caption{
        Gráfico corrente x tensão de um diodo zener \\ \\
        \tiny
        Por Filip Dominec - Obra do próprio, CC BY-SA 3.0, https://commons.wikimedia.org/w/index.php?curid=1440010
    }
\end{figure}

\section{Parte Prática}

\subsection{Questão 3}

\begin{tcolorbox}[title=\large Questão 3, colback=red!5!white, colframe=red!75!black]
    \large
    alimentar o circuito e medir os valores da corrente e tensão média na carga
\end{tcolorbox}

Os seguintes valores foram medidos durante o experiemento:

\begin{center}
    \begin{tcolorbox}[width=0.6\textwidth]
        \centering
        $I_{DC} = 135mA$
        \hspace{2cm}
        $V_{DC} = 28,86V$
    \end{tcolorbox}
\end{center}

\subsection{Questão 4}

\begin{tcolorbox}[title=\large Questão 4, colback=red!5!white, colframe=red!75!black]
    \large
    utilizando os valores medidos no item anterior calcular o valor da resistência de carga e comparar com o valor indicado no módulo Fonte CC. Explicar possíveis diferenças;
\end{tcolorbox}

Valor de resistência encontrado:
    
\begin{center}
    \begin{tcolorbox}[width=0.3\textwidth]
        \centering
        $R_{C} = 213,77\Omega$
    \end{tcolorbox}
\end{center}

\textbf{Resposta}: Essa diferença pode ser devida às tolerâncias dos componentes ou à precisão do equipamento de medição. A resistência de um componente pode variar dentro de uma determinada percentagem do seu valor nominal, que é representado pela série de tolerâncias do componente. Além disso, as condições ambientais, como a temperatura, podem afetar a resistência medida.

\subsection{Questão 8}

\begin{tcolorbox}[title=\large Questão 8, colback=red!5!white, colframe=red!75!black]
    \large
    desenhar as formas de onda das tensões no capacitor C3 e na carga no eixo ao lado. Comentar as formas de onda obtidas;
\end{tcolorbox}

\begin{figure}[h!]
    \centering
    \includegraphics[width=12cm]{images/SCR06.PNG}
    \caption{Osciloscópio, primeira montagem}
\end{figure}

\subsection{Questão 12}

\begin{tcolorbox}[title=\large Questão 12, colback=red!5!white, colframe=red!75!black]
    \large
    analisar em função do fator de ripple na carga qual o efeito da inclusão do diodo zener no circuito retificador;
\end{tcolorbox}

\textbf{Resposta}: A inclusão do diodo Zener no circuito retificador serve para estabilizar a tensão de saída, reduzindo o ripple. O diodo Zener mantém a tensão sobre a carga constante ao seu valor de Zener, mesmo com flutuações na tensão de entrada ou carga.

\subsection{Questão 13}

\begin{tcolorbox}[title=\large Questão 13, colback=red!5!white, colframe=red!75!black]
    \large
    ligar agora o canal 1 do osciloscópio para observar a forma de onda da corrente no diodo zener. Para isso ligar os terminais do osciloscópio no resistor R3 de 1$\Omega$. Como o valor do resistor é conhecido, basta dividir o valor do fator de escala do osciloscópio (Volts/divisão) por $1\Omega$ para que se tenha um fator de escala graduado em Ampère/divisão. Fazer isso e anotar o fator de escala encontrado;
\end{tcolorbox}

\begin{center}
    \begin{tcolorbox}[width=0.4\textwidth, title=\large Fator de escala em Ampere/div]
        \centering
        $50mA$
    \end{tcolorbox}
\end{center}

\subsection{Questão 14}

\begin{tcolorbox}[title=\large Questão 14, colback=red!5!white, colframe=red!75!black]
    \large
    observar a forma de onda na carga e em R3 simultaneamente no osciloscópio. Tire conclusões sobre a condução do diodo. Desenhar as formas de onda da corrente IZ e da tensão na carga no eixo ao lado. Anotar os dois fatores de escala;
\end{tcolorbox}

\begin{figure}[h!]
    \centering
    \includegraphics[width=12cm]{images/SCR07.PNG}
    \caption{Osciloscópio, segunda montagem (capacitor C3)}
\end{figure}

\subsection{Questão 17}

\begin{tcolorbox}[title=\large Questão 17, colback=red!5!white, colframe=red!75!black]
    \large
    religar o módulo Fonte CC e medir a corrente e tensão na carga;
\end{tcolorbox}

Os seguintes valores foram medidos durante o experiemento:

\begin{center}
    \begin{tcolorbox}[width=0.6\textwidth]
        \centering
        $I_{RL} = 60mA$
        \hspace{2cm}
        $V_{RL} = 12,88V$
    \end{tcolorbox}
\end{center}

\subsection{Questão 19}

\begin{tcolorbox}[title=\large Questão 19, colback=red!5!white, colframe=red!75!black]
    \large
    desenhar as formas de onda da corrente $I_Z$ e da tensão na carga no eixo ao lado. Anotar os dois fatores de escala. Tire conclusões sobre a condução do diodo.
\end{tcolorbox}

\begin{center}
    \begin{tcolorbox}[width=0.4\textwidth, title=\large Fatores de escala (Ampere/div)]
        \centering
        $I_{In} = 50mA$
        \hspace{2cm}
        $I_{Z} = 100mA$
    \end{tcolorbox}
\end{center}

\begin{figure}[h!]
    \centering
    \includegraphics[width=12cm]{images/SCR08.PNG}
    \caption{Osciloscópio, segunda montagem (capacitor C1)}
\end{figure}

\subsection{Questão 20}

\begin{tcolorbox}[title=\large Questão 20, colback=red!5!white, colframe=red!75!black]
    \large
    medir a corrente e tensão na carga;
\end{tcolorbox}

Os seguintes valores foram medidos durante o experiemento:

\begin{center}
    \begin{tcolorbox}[width=0.6\textwidth]
        \centering
        $I_{RL} = 50,5mA$
        \hspace{2cm}
        $V_{RL} = 10,84V$
    \end{tcolorbox}
\end{center}

\section{Questionário}

\subsection{Questão 1}

\begin{tcolorbox}[title=\large Questão 1, colback=red!5!white, colframe=red!75!black]
    \large
    qual a limitação de uso do diodo zener como regulador no circuito estudado?
\end{tcolorbox}

O diodo Zener tem uma limitação de dissipação de potência; ele só pode dissipar uma quantidade finita de energia na forma de calor antes de ser danificado. Se a corrente que passa por ele for muito alta, ele pode superaquecer e falhar. Portanto, o diodo Zener no circuito é limitado pela sua capacidade máxima de corrente e pela potência que pode dissipar de forma segura.

\subsection{Questão 2}

\begin{tcolorbox}[title=\large Questão 2, colback=red!5!white, colframe=red!75!black]
    \large
    pesquisar na Internet alguns fabricantes e modelos de diodos zener com tensão de 5,1V, 8,2V, 9V e 12V
\end{tcolorbox}

\textbf{UNIT Electronics}:
\begin{itemize}
\large
    \item 1N4733A: Diodo Zener de 5,1V
    \item 1N4738A: Diodo Zener de 8,2V.
    \item 1N4742A: Diodo Zener de 12V
\end{itemize}

\vspace{\baselineskip}

\textbf{Micro Commercial Co (DigiKey)}:
\begin{itemize}
\large
    \item 3SMAJ5918B-TP: para 5.1V 
    \item 3SMAJ5923B-TP: para 8.2V
\end{itemize}

\vspace{\baselineskip}


\textbf{Vishay Semiconductors}:
\begin{itemize}
\large
    \item PLZ9V1A-G3/H: para 9v
\end{itemize}

\subsection{Questão 3}

\begin{tcolorbox}[title=\large Questão 3, colback=red!5!white, colframe=red!75!black]
    \large
    desenhar a curva característica do diodo zener mostrando todas as correntes e tensões de interesse.
\end{tcolorbox}

\subsection{Questão 4}

\begin{tcolorbox}[title=\large Questão 4, colback=red!5!white, colframe=red!75!black]
    \large
    explicar a função da ponte retificadora, do capacitor e do resistor $R_2$ no circuito mostrado na Figura 10.
\end{tcolorbox}

A ponte retificadora transforma a corrente alternada em contínua pulsante. O capacitor suaviza a tensão pulsante, reduzindo o ripple. O resistor $R_2$ limita a corrente para o diodo Zener, protegendo-o de sobrecorrente

\end{document}